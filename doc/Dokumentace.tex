\documentclass[12pt, a4paper]{article}
\usepackage[utf8]{inputenc}
\usepackage[IL2]{fontenc}
\usepackage[czech]{babel}

\usepackage[pdftex]{hyperref}
\hypersetup{colorlinks=true,
  unicode=true,
  linkcolor=black,
  citecolor=black,
  urlcolor=black,
  bookmarksopen=true}
\usepackage{graphicx}
\usepackage{caption}
\usepackage{url}


%Nastavi hloubku obsahu \setcounter{tocdepth}{3}

\begin{document}
	\begin{titlepage}
		\begin{center}
			\includegraphics{img/ZCULogo.pdf}\\[1cm]
			\textsc{\LARGE Západočeská univerzita v~Plzni}\\[0.1cm]
			\textsc{\Large Fakulta aplikovaných věd}\\[0.1cm]
			\textsc{\large Katedra informatiky a~výpočetní techniky}
			\vfill
			\textsc{\LARGE Semestrální práce KIV/MKZ}\\[0.2cm]
			\LARGE{Studentův pomocník}
			\vfill
			Jaroslav Klaus\\[0.2cm]
			\today, Plzeň
		\end{center}
	\end{titlepage}

	\tableofcontents
	\newpage

	\section{Zadání}
	V rámci samostatné práce je potřeba vytvořit předem odsouhlasenou aplikaci na zvolené cílové platformě, předvést funkčnost aplikace v emulátoru nebo na fyzickém zařízení a odevzdat okomentovaný kód práce včetně dokumentace.
	
	Pro splnění těchto podmínek jsem si zvolil tvorbu aplikace pro platformu Android, která by měla sloužit jako pomocník pro studenta vysoké školy, na které je používán informační systém Stag. Aplikace bude zobrazovat rozvrh studenta získaný ze Stagu po dnech a dovolí mu si ke každému z~předmětů zobrazit či přidat další informace, jako jsou
		\begin{itemize}
			\item sylabus předmětu,
			\item podmínky pro splnění předmětu,
			\item úkoly, které je potřeba udělat
			\item a~počet absencí v~předmětu.
		\end{itemize}
	
	\section{Programátorská dokumentace}
	
	
	\section{Uživatelská dokumentace}
	
	\section{Řešené problémy}
	
	\section{Testování}
	
	\section{Závěr}
	
	
	\newpage
	\begin{thebibliography}{9}
		
	\end{thebibliography}
	
\end{document}